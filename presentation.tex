\documentclass{beamer}
\usepackage{authblk}
\usepackage{amsmath, amssymb, amsthm}
\usepackage{braket}
\usepackage{algorithm}
\usepackage{algpseudocode}
\newcommand{\sgn}{\operatorname{sgn}}
\DeclareMathOperator*{\E}{\mathbb{E}}
\DeclareMathOperator*{\spn}{\operatorname{span}}
\DeclareMathOperator*{\poly}{\operatorname{poly}}
\newcommand{\norm}[1]{\left\lVert#1\right\rVert}
\usepackage{etoolbox}
\usepackage{hyperref}
\newcommand{\algorithmautorefname}{Algorithm}
\apptocmd{\thebibliography}{\raggedright}{}{}
\usepackage{cite,color,float}
\newcommand{\abs}[1]{\lvert#1\rvert}

\usetheme{Copenhagen}

\title{Delegation of Quantum Computations}
\subtitle{Based on Local Hamiltonians}

\begin{document}

\frame{
	\titlepage
}

\frame{
	\frametitle{Settings}
	\begin{itemize}[<+->]
		\item BPP client, BQP server
		\item Client has $C=U_n\ldots U_1$ and $x$, wants to compute $C\ket{x}$
		\item Possible goals are the following:
		\begin{itemize}
			\item \color<7->{red}{Verifiability}
			\item Privacy
			\item Long output: Not as trivial as it seems!
		\end{itemize}
	\end{itemize}
}

\frame{
	\frametitle{BQNP (QMA)}
	\begin{itemize}[<+->]
		\item Quantum analogue of NP or MA (Merlin-Arthur)
		\item Amplification can be done but is nontrivial.
		\item Certificate can be quantum
		\item {\color<+->{gray}{Verifier is a BQP machine}}
		\item Verifier can store qubits and apply X/Z measurements.
	\end{itemize}
}

\frame{
	\frametitle{Local Hamiltonian}
	\begin{itemize}[<+->]
		\item Is QMA-complete
		\item Given the following input:
			\begin{itemize}
				\item $\set{H_j}$ each Hermitian and acts on at most $k$ qubits.
				\item $a, b\in\mathbb{R}$; $b>a$ with at least inverse polynomial gap
			\end{itemize}
		\item Promise problem; decide which of the following is true about $H=\sum H_j$
			\begin{itemize}
				\item $H$ has an eigenvalue not exceeding $a$
				\item All eigenvalues of $H$ are greater than $b$
			\end{itemize}
		\item Special case: $H_j\in\mathcal{G}_{XZ}=\set{U_0\otimes U_1\otimes\ldots\otimes U_n: U_i\in\set{I,X,Z}}$
	\end{itemize}
}
\frame{
	\frametitle{Local Hamiltonian is in QMA}
	Goal: Estimate
	$$\braket{\phi|H|\phi}=\sum_{S\in\mathcal{G}_{XZ}}d_S S$$
	\begin{align*}
		\onslide<2->{\frac{1}{\sum_{S\in\mathcal{G}_{XZ}}\abs{d_S}}\braket{\phi|H|\phi}&
			=\sum_{S\in\mathcal{G}_{XZ}} p_S\sgn(d_S)\braket{\phi|S|\phi}\\}
		\onslide<3->{&=\sum_{S\in\mathcal{G}_{XZ}} p_S\sgn(d_S)\E[\lambda_S]\\}
		\onslide<4->{&=\E_S[\sgn(d_S)\E[\lambda_S]]\\}
		\onslide<5->{&=\E_S[\sgn(d_S)\lambda_S]}
	\end{align*}
}

\frame{
	\frametitle{Local Hamiltonian is in QMA cont.}
	Let $p=P[\sgn(d_S)\lambda_S=-1]$
	\pause
	$$\Rightarrow \frac{1}{D}\braket{\phi|H|\phi}=\E_S[\sgn(d_S)\lambda_S]=-p+(1-p)$$
	\pause
	$$\Rightarrow p=\frac{1}{2}-\frac{1}{2D}\braket{\phi|H|\phi}$$
}

\frame{
	\frametitle{Preliminaries}
	\begin{itemize}
		\item Hadamard and Toffoli with real states is universal
		\item $H, T\in\spn\mathcal{G}_{XZ}$
	\end{itemize}
}

\frame{
	\frametitle{Local Hamiltonian is QBP-hard}
	Let $C=U_T\ldots U_1$ be a circuit where $U_i\in\set{H, T}$, and $\ket{x}$ be the initial state.
	\pause
	Construct $H$ so
	$$\ket{\phi}=\sum_{t=0}^TU_t\ldots U_1\ket{x}\otimes\ket{\hat{t}}$$
	has low eigenvalue, while everything perpendicular to it has high eigenvalue.
}

\frame{
	\frametitle{Checking for valid outputs}
	$$H_{out}=(I-\ket{1}\bra{1}_0)\otimes\ket{\hat{T}}\bra{\hat{T}}$$
	\pause
	$$H_{out}\in\spn\mathcal{G}_{XZ}$$
	$$H_{out}=\left(\frac{1}{2}(I-Z_1)\right)\otimes\left(\frac{1}{2}(1+Z_T)\right)$$
}

\frame{
	\frametitle{Checking for valid inputs}
	$$H_{in}=\sum_{i=1}^n(I-\ket{x_i}\bra{x_i})\otimes\ket{0}\bra{0}_1$$
	\pause
	$$=\sum_{i=1}^n\left(\frac{1}{2}I-(-1)^{x_i}Z_i\right)\otimes\left(\frac{1}{2}(I+Z_1)\right)$$
}

\frame{
	\frametitle{Checking for legal clock states}
	$$H_{clock}=\sum_{t=1}^{T-1}\ket{01}\bra{01}_{t,t+1}$$
	\pause
	$$=\frac{1}{4}(Z_1 - Z_T) + \frac{1}{4}\sum_{t=1}^{T-1}(I-Z_tZ_{t+1})$$
}

\frame{
	\frametitle{Checking for valid propagation}
	$$H_{prop}=\sum_{t\in T_1}H_{prop,t}$$
	\pause
	Intuitively,
	$$I\otimes\ket{\widehat{t}}\bra{\widehat{t}}
	-U_t\otimes\ket{\widehat{t}}\bra{\widehat{t-1}}$$
	\pause
	$$H_{prop,t}=I\otimes\ket{\widehat{t}}\bra{\widehat{t}}
	+I\otimes\ket{\widehat{t-1}}\bra{\widehat{t-1}}
	-U_t\otimes\ket{\widehat{t}}\bra{\widehat{t-1}}
	-U_t^\dagger\otimes\ket{\widehat{t-1}}\bra{\widehat{t}}$$
}

\frame{
	\frametitle{$H_{prop}\in\spn\mathcal{G}_{XZ}$}
	$$H_{prop,t}=I\otimes\ket{\widehat{t}}\bra{\widehat{t}}
	+I\otimes\ket{\widehat{t-1}}\bra{\widehat{t-1}}
	-U_t\otimes\ket{\widehat{t}}\bra{\widehat{t-1}}
	-U_t^\dagger\otimes\ket{\widehat{t-1}}\bra{\widehat{t}}$$
	$$=\frac{I}{4}\otimes(I-Z_{t-1})(I+Z_{t+1})-\frac{U_t}{4}\otimes(I-Z_{t-1})X_t(I+Z_{t+1})$$
	\pause
	$$H_{prop,1}=\frac{1}{2}(I+Z_2)-U_1\otimes\frac{1}{2}(X_1+X_1Z_2)$$
	$$H_{prop,T}=\frac{1}{2}(I-Z_{t-1})-U_T\otimes\frac{1}{2}(X_T-Z_{T-1}X_T)$$
}

\frame{
	\frametitle{Projection Lemma}
	Let $H_1, H_2$ be Hamiltonians where $H_2\geq0$, and $S=\ker H_2$.
	Then $\exists J\in\mathbb{R}$ s.t.
	$\lambda(H_1\big|_S)-\frac{1}{8}\leq
		\lambda(H_1+JH_2)\leq\lambda(H_1\big|_S)$.
	\pause
	\begin{block}{idea}
		Construct $H_2$ to ``cut out" vectors outside $\ker H_2$
	\end{block}
}

\frame{
	\frametitle{Analysis of the above}
	We get
	$$H=H_{out}+J_{in}H_{in}+J_{prop}H_{prop}+J_{clock}H_{clock}$$
	by applying the projection lemma iteratively.

	Then we have $\lambda(H)\geq\lambda(H\big|_K)-\frac{3}{8}$

	
	TODO Prove the above Hamiltonians are in fact psd and have lower bound on eigenvalue?
}

\frame{
	\frametitle{Conclusion}
	By the projection lemma,
	$$\lambda(H_{out}\big|_{S_{in}}-\frac{4}{8})\leq
	\lambda(H)\leq\lambda(H_{out}\big|_{S_{in}})$$
	\pause
	\begin{itemize}[<+->]
	\item If the circuit rejects with probability less than $\varepsilon$,
	then $\lambda(H)<\varepsilon$.
	\item Otherwise the circuit should reject with probability
	greater than $1-\varepsilon$. Which means
	$\lambda(H)>1-\varepsilon-\frac{4}{8}=\frac{1}{2}-\varepsilon$.
	\item So 2-local Hamiltonian is QMA-hard.
	\end{itemize}
}

\iffalse
\frame{
	\frametitle{General case: Local Hamiltonian is in QMA}

}
\fi

\end{document}
