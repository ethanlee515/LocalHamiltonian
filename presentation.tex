\documentclass{beamer}
\usepackage{authblk}
\usepackage{amsmath, amssymb, amsthm}
\usepackage{braket}
\usepackage{algorithm}
\usepackage{algpseudocode}
\newcommand{\sgn}{\operatorname{sgn}}
\DeclareMathOperator*{\E}{\mathbb{E}}
\DeclareMathOperator*{\spn}{\operatorname{span}}
\DeclareMathOperator*{\poly}{\operatorname{poly}}
\newcommand{\norm}[1]{\left\lVert#1\right\rVert}
\usepackage{etoolbox}
\usepackage{hyperref}
\newcommand{\algorithmautorefname}{Algorithm}
\apptocmd{\thebibliography}{\raggedright}{}{}
\usepackage{cite,color,float}

\usetheme{Copenhagen}

\title{Delegation of Quantum Computations}
\subtitle{Based on Local Hamiltonians}

\begin{document}

\frame{
	\titlepage
}

\frame{
	\frametitle{Settings}
	\begin{itemize}[<+->]
		\item BPP client, BQP server
		\item Client has $C=U_n\ldots U_1$ and $x$, wants to compute $Cx$
		\item Verifiability (TODO define)
		\item Privacy (TODO define)
		\item Long output: Not as trivial as it seems!
	\end{itemize}
}

\frame{
	\frametitle{BQNP (QMA)}
	\begin{itemize}[<+->]
		\item Quantum analogue of NP or MA (Merlin-Arthur)
		\item Certificate can be quantum
		\item {\color<+->{gray}{Verifier is a BQP machine}}
		\item Verifier can store qubits and apply X/Z measurements.
	\end{itemize}
}

\frame{
	\frametitle{Local Hamiltonian}
	\begin{itemize}
		\item Is QMA-hard
		\item Given
			\begin{itemize}
			\item $H=\sum_j H_j$, where $H_j$ is Hermitian and acts on at most $k$ qubits.
			\item $a, b\in\mathbb{R}$ s.t. $b>a$ with at least inverse polynomial gap
			\end{itemize}
		\item Promise problem; decide which of the following is true
			\begin{itemize}
			\item $H$ has an eigenvalue not exceeding $a$
			\item All eigenvalues of $H$ are greater than $b$
			\end{itemize}
		\item Generalization: $H_j\in\mathcal{G}_{XZ}=\set{U_0\otimes U_1\otimes\ldots\otimes U_n: U_i\in\set{I,X,Z}}$
	\end{itemize}
}

\frame{
	\frametitle{Notation}
	\begin{itemize}
		\item Let $\lambda(H)$ denote the smallest eigenvalue of $H$.
		\item Let $H\big|_S=\prod_SH\prod_S$.
		\item Let $\widehat{t}$ denote $t$ represented in unary.
	\end{itemize}
}

\frame{
	\frametitle{Change of definition}

	An operator $H:\mathcal{B}^{\otimes n}\rightarrow \mathcal{B}^{\otimes n}$
	is a $k$-local Hamiltonian if $H=\sum_{j=1}^rH_j$
	where $H_i$ are hermitian acting on at most $k$ qubits.

	\pause
	\begin{block}{note}
		The positive semidefinite condition is no longer there.
	\end{block}
}

\frame{
	\frametitle{Assumptions}
	\begin{itemize}[<+->]
		\item The circuit contains only controlled phase gate $C_\phi$
			and one-qubit gates
		\item $C_\phi$ gates are preceded and followed by two $Z$ gates
		\item $C_\phi$ gates are applied at $t\in T_2=\{L, 2L, 3L,...\}$.
		\item There are $T=(\left|T_2\right|+1)L-1$ gates in total.
		\item Let $T_1=[T]\setminus T_2$ be
			time corresponding to 1-qubit gates
	\end{itemize}
}

\frame{
	\frametitle{Projection Lemma}

	Let $H_1, H_2$ be Hamiltonians where $H_2\geq0$, and $S=\ker H_2$.
	Then $\exists J\in\mathbb{R}$ s.t.
	$\lambda(H_1\big|_S)-\frac{1}{8}\leq
		\lambda(H_1+JH_2)\leq\lambda(H_1\big|_S)$.

	\pause
	\begin{block}{idea}
		Construct $H_2$ to ``cut out" vectors outside $\ker H_2$
	\end{block}
}

\frame{
	\frametitle{2-local construction overview}
	$$H=H_{out}+J_{in}H_{in}+J_2H_{prop2}+J_1H_{prop1}+J_{clock}H_{clock}$$
	\pause
	\begin{itemize}[<+->]
		\item $H_{clock}$ corresponds to valid (unary) clock states
		\item $H_{prop1}$ corresponds to correct propagations
			of 1-qubit gates
		\item $H_{prop}$ corresponds to correct propagation
		\item $H_{in}$ corresponds to the input qubits
			being initialized properly
		\item $S_{legal}\supset S_{prop1}\supset S_{prop}\supset S_{in}$
	\end{itemize}
}

\frame{
	\frametitle{Restriction to legal clock states}
	$$H_{clock}=\sum_{1\leq i<j\leq T} I\otimes |01\rangle\langle01|_{ij}$$
	\pause
	When restricted to $S_{legal}=\ker H_{clock}$:
	\begin{itemize}[<+->]
		\item $|10\rangle\langle10|_{t,t+1}
			=|\widehat{t}\rangle\langle\widehat{t}|$ 
		\item $|1\rangle\langle0|_{t+1}
			=|\widehat{t+1}\rangle\langle\widehat{t}|$
		\item $|11\rangle\langle00|_{t,t+1}
			=|\widehat{t+2}\rangle\langle\widehat{t}|$
	\end{itemize}
}

\frame{
	\frametitle{Restriction to $S_{prop1}$}
	$$H_{prop1}=\sum_{t\in T_1}H_{prop,t}$$
	$$H_{prop,t}=I\otimes|\widehat{t}\rangle\langle\widehat{t}|
		+I\otimes|\widehat{t-1}\rangle\langle\widehat{t-1}|
		-U_t\otimes|\widehat{t}\rangle\langle\widehat{t-1}|
		-U_t^\dagger\otimes|\widehat{t-1}\rangle\langle\widehat{t}|$$
}

\iffalse
\frame{
	\frametitle{Restriction to $S_{prop1}$ cont.}
	Let $t\in T_2$. Then vectors in $S_{prop1}$ can be written as linear
		combinations of the following:
	\begin{itemize}
		\item $|00\rangle_{f_t,s_t}|\xi_{00}\rangle\otimes
			(|\widehat{t}\rangle+|\widehat{t+1}\rangle+|\widehat{t+2}\rangle)$
		\item $|01\rangle_{f_t,s_t}|\xi_{01}\rangle\otimes
			(|\widehat{t}\rangle+|\widehat{t+1}\rangle-|\widehat{t+2}\rangle)$
		\item $|10\rangle_{f_t,s_t}|\xi_{10}\rangle\otimes
			(|\widehat{t}\rangle-|\widehat{t+1}\rangle-|\widehat{t+2}\rangle)$
		\item $|11\rangle_{f_t,s_t}|\xi_{11}\rangle\otimes
			(|\widehat{t}\rangle-|\widehat{t+1}\rangle+|\widehat{t+2}\rangle)$
	\end{itemize}
}
\fi

\frame{
	\frametitle{Restriction to $S_{prop1}$ cont.}
	Let $t\in T_2$. Then the following are true:
	\begin{itemize}[<+->]
		\item $|0\rangle\langle0|_{f_t}\otimes
			|\widehat{t}\rangle\langle\widehat{t}|
			=(|\widehat{t}\rangle+|\widehat{t+1}\rangle)
			(\langle\widehat{t}|+\langle\widehat{t+1}|)$
		\item $|1\rangle\langle1|_{f_t}\otimes
			|\widehat{t}\rangle\langle\widehat{t}|
			=(|\widehat{t}\rangle-|\widehat{t+1}\rangle)
			(\langle\widehat{t}|-\langle\widehat{t+1}|)$
		\item $|0\rangle\langle0|_{s_t}\otimes
			|\widehat{t}\rangle\langle\widehat{t}|
			=(|\widehat{t+1}\rangle+|\widehat{t+2}\rangle)
			(\langle\widehat{t+1}|+\langle\widehat{t+2}|)$
		\item $|1\rangle\langle1|_{s_t}\otimes
			|\widehat{t}\rangle\langle\widehat{t}|
			=(|\widehat{t+1}\rangle-|\widehat{t+2}\rangle)
			(\langle\widehat{t+1}|-\langle\widehat{t+2}|)$
		\item $(|00\rangle\langle00|_{f_t,s_t}
			+|11\rangle\langle11|_{f_t,s_t})\otimes
			|\widehat{t}\rangle\langle\widehat{t}|
			=(|\widehat{t}\rangle+|\widehat{t+2}\rangle)
			(\langle\widehat{t}|+\langle\widehat{t+2}|)$
		\item $(|01\rangle\langle01|_{f_t,s_t}
			+|10\rangle\langle10|_{f_t,s_t})\otimes
			|\widehat{t}\rangle\langle\widehat{t}|
			=(|\widehat{t}\rangle-|\widehat{t+2}\rangle)
			(\langle\widehat{t}|-\langle\widehat{t+2}|)$
	\end{itemize}
}

\frame{
	\frametitle{Restriction to $S_{prop1}$ cont.}
	Similarly,
	\begin{itemize}
		\item $|0\rangle\langle0|_{f_t}\otimes
			|\widehat{t-1}\rangle\langle\widehat{t-1}|
			=(|\widehat{t-1}\rangle+|\widehat{t-2}\rangle)
			(\langle\widehat{t-1}|+\langle\widehat{t-2}|)$
		\item $|1\rangle\langle1|_{f_t}\otimes
			|\widehat{t-1}\rangle\langle\widehat{t-1}|
			=(|\widehat{t-1}\rangle-|\widehat{t-2}\rangle)
			(\langle\widehat{t-1}|-\langle\widehat{t-2}|)$
		\item $|0\rangle\langle0|_{s_t}\otimes
			|\widehat{t-1}\rangle\langle\widehat{t-1}|
			=(|\widehat{t-2}\rangle+|\widehat{t-3}\rangle)
			(\langle\widehat{t-2}|+\langle\widehat{t-3}|)$
		\item $|1\rangle\langle1|_{s_t}\otimes
			|\widehat{t-1}\rangle\langle\widehat{t-1}|
			=(|\widehat{t-2}\rangle-|\widehat{t-3}\rangle)
			(\langle\widehat{t-2}|-\langle\widehat{t-3}|)$
		\item $(|00\rangle\langle00|_{f_t,s_t}
			+|11\rangle\langle11|_{f_t,s_t})\otimes
			|\widehat{t-1}\rangle\langle\widehat{t-1}|
			=(|\widehat{t-1}\rangle+|\widehat{t-3}\rangle)
			(\langle\widehat{t-1}|+\langle\widehat{t-3}|)$
		\item $(|01\rangle\langle01|_{f_t,s_t}
			+|10\rangle\langle10|_{f_t,s_t})\otimes
			|\widehat{t-1}\rangle\langle\widehat{t-1}|
			=(|\widehat{t-1}\rangle-|\widehat{t-3}\rangle)
			(\langle\widehat{t-1}|-\langle\widehat{t-3}|)$
	\end{itemize}
}

\frame{
	\frametitle{Restriction to $S_{prop}$}
	Naive construction:
	$$H_t=I\otimes|\widehat t\rangle\langle\widehat t|
		+I\otimes|\widehat{t-1}\rangle\langle\widehat{t-1}|
		-C_\phi\otimes|\widehat t\rangle\langle\widehat{t-1}|
		-C_\phi^\dagger\otimes|\widehat{t-1}\rangle\langle\widehat{t}|$$
	$$H'=\sum_{t\in T_2}H_t$$
	\pause
	Equivalently, $H_t=$

	$|00\rangle\langle00|_{f_t,s_t}\otimes
		(|\widehat{t}\rangle-|\widehat{t-1}\rangle)
		(\langle\widehat{t}|-\langle\widehat{t-1}|)+$

	$|01\rangle\langle01|_{f_t,s_t}\otimes
		(|\widehat{t}\rangle-|\widehat{t-1}\rangle)
		(\langle\widehat{t}|-\langle\widehat{t-1}|)+$

	$|10\rangle\langle10|_{f_t,s_t}\otimes
		(|\widehat{t}\rangle-|\widehat{t-1}\rangle)
		(\langle\widehat{t}|-\langle\widehat{t-1}|)+$

	$|11\rangle\langle11|_{f_t,s_t}\otimes
		(|\widehat{t}\rangle+|\widehat{t-1}\rangle)
		(\langle\widehat{t}|+\langle\widehat{t-1}|)$
}

\iffalse

\frame{
	\frametitle{Restriction to $S_{prop}$ cont.}
	$S_{prop1}=\ker H_{prop1}$ is spanned by the following:
	$$|\eta_{l,i}\rangle=\frac{1}{\sqrt{L}}
		\sum_{t=lL}^{(l+1)L-1}
		U_t...U_1|i\rangle\otimes|\widehat{t}\rangle$$
	\pause
	Then,
	$$H'=\frac{1}{L}\sum_{i=0}^{2^N-1}\sum_{l=1}^{T_2}
		(|\eta_{l-1,i}\rangle-|\eta_{l,i}\rangle)
		(\langle\eta_{l-1,i}|-\langle\eta_{l,i}|)$$
}

\frame{
	\frametitle{Restriction to $S_{prop}$ cont.}
	PROOF TODO
}

\fi

\frame{
	\frametitle{Restriction to $S_{prop}$ cont.}
	$$H_{prop2}=\sum_{t\in T_2}H_{prop2,t}$$

	$H_{prop2,t}=$

	$|00\rangle\langle00|_{f_t,s_t}\otimes
		4(|\widehat{t}\rangle-|\widehat{t-1}\rangle)
		(\langle\widehat{t}|-\langle\widehat{t-1}|)+$

	$|01\rangle\langle01|_{f_t,s_t}\otimes
		2(|\widehat{t}\rangle-|\widehat{t-1}\rangle)
		(\langle\widehat{t}|-\langle\widehat{t-1}|)+$

	$|10\rangle\langle10|_{f_t,s_t}\otimes
		2(|\widehat{t}\rangle-|\widehat{t-1}\rangle)
		(\langle\widehat{t}|-\langle\widehat{t-1}|)+$

	$|11\rangle\langle11|_{f_t,s_t}\otimes
		(|\widehat{t}\rangle+|\widehat{t-1}\rangle)
		(\langle\widehat{t}|+\langle\widehat{t-1}|)$
}

\frame{
	\frametitle{Restriction to $S_{prop}$ cont.}

	$H_{prop2,t}=H_{qubit,t}+H_{time,t}=$
	\pause
	$$(-2|0\rangle\langle0|_{f_t}-2|0\rangle\langle0|_{s_t}
		+|1\rangle\langle1|_{f_t}+|1\rangle\langle1|_{s_t})\otimes$$
	$$(|\widehat{t}\rangle\langle\widehat{t-1}|+
		|\widehat{t-1}\rangle\langle\widehat{t}|)+$$
	\pause
	$$(2|0\rangle\langle0|_{f_t}+2|0\rangle\langle0|_{s_t}
		+|1\rangle\langle1|_{f_t}+|1\rangle\langle1|_{s_t}
		-2|01\rangle\langle01|_{f_t,s_t}
		-2|10\rangle\langle10|_{f_t,s_t})\otimes$$
	$$(|\widehat{t-1}\rangle\langle\widehat{t-1}|+
		|\widehat{t}\rangle\langle\widehat{t}|)$$
}

\frame{
	\frametitle{Restriction to $S_{in}$}
	$$H_{in}=\sum_{i=m+1}^N|1\rangle\langle1|_i\otimes|0\rangle\langle0|$$
	$$H_{out}=(T+1)|0\rangle\langle0|_1\otimes|T\rangle\langle T|$$
	Then $\langle\eta|H_{out}|\eta\rangle$ is the eigenvalue,
	as well the probability of rejection.
}

\frame{
	\frametitle{Conclusion}
	By the projection lemma,
	$$\lambda(H_{out}\big|_{S_{in}}-\frac{4}{8})\leq
	\lambda(H)\leq\lambda(H_{out}\big|_{S_{in}})$$

	\pause
	\begin{itemize}[<+->]
	\item If the circuit rejects with probability less than $\varepsilon$,
	then $\lambda(H)<\varepsilon$.
	\item Otherwise the circuit should reject with probability
	greater than $1-\varepsilon$. Which means
	$\lambda(H)>1-\varepsilon-\frac{4}{8}=\frac{1}{2}-\varepsilon$.
	\item So 2-local Hamiltonian is QMA-hard.
	\end{itemize}
}

\end{document}
